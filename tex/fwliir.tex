
    




    
\documentclass[11pt]{article}

    
    \usepackage[breakable]{tcolorbox}
    \tcbset{nobeforeafter} % prevents tcolorboxes being placing in paragraphs
    \usepackage{float}
    \floatplacement{figure}{H} % forces figures to be placed at the correct location
    
    \usepackage[T1]{fontenc}
    % Nicer default font (+ math font) than Computer Modern for most use cases
    \usepackage{mathpazo}

    % Basic figure setup, for now with no caption control since it's done
    % automatically by Pandoc (which extracts ![](path) syntax from Markdown).
    \usepackage{graphicx}
    % We will generate all images so they have a width \maxwidth. This means
    % that they will get their normal width if they fit onto the page, but
    % are scaled down if they would overflow the margins.
    \makeatletter
    \def\maxwidth{\ifdim\Gin@nat@width>\linewidth\linewidth
    \else\Gin@nat@width\fi}
    \makeatother
    \let\Oldincludegraphics\includegraphics
    % Set max figure width to be 80% of text width, for now hardcoded.
    \renewcommand{\includegraphics}[1]{\Oldincludegraphics[width=.8\maxwidth]{#1}}
    % Ensure that by default, figures have no caption (until we provide a
    % proper Figure object with a Caption API and a way to capture that
    % in the conversion process - todo).
    \usepackage{caption}
    \DeclareCaptionLabelFormat{nolabel}{}
    \captionsetup{labelformat=nolabel}

    \usepackage{adjustbox} % Used to constrain images to a maximum size 
    \usepackage{xcolor} % Allow colors to be defined
    \usepackage{enumerate} % Needed for markdown enumerations to work
    \usepackage{geometry} % Used to adjust the document margins
    \usepackage{amsmath} % Equations
    \usepackage{amssymb} % Equations
    \usepackage{textcomp} % defines textquotesingle
    % Hack from http://tex.stackexchange.com/a/47451/13684:
    \AtBeginDocument{%
        \def\PYZsq{\textquotesingle}% Upright quotes in Pygmentized code
    }
    \usepackage{upquote} % Upright quotes for verbatim code
    \usepackage{eurosym} % defines \euro
    \usepackage[mathletters]{ucs} % Extended unicode (utf-8) support
    \usepackage[utf8x]{inputenc} % Allow utf-8 characters in the tex document
    \usepackage{fancyvrb} % verbatim replacement that allows latex
    \usepackage{grffile} % extends the file name processing of package graphics 
                         % to support a larger range 
    % The hyperref package gives us a pdf with properly built
    % internal navigation ('pdf bookmarks' for the table of contents,
    % internal cross-reference links, web links for URLs, etc.)
    \usepackage{hyperref}
    \usepackage{longtable} % longtable support required by pandoc >1.10
    \usepackage{booktabs}  % table support for pandoc > 1.12.2
    \usepackage[inline]{enumitem} % IRkernel/repr support (it uses the enumerate* environment)
    \usepackage[normalem]{ulem} % ulem is needed to support strikethroughs (\sout)
                                % normalem makes italics be italics, not underlines
    \usepackage{mathrsfs}
    

    
    % Colors for the hyperref package
    \definecolor{urlcolor}{rgb}{0,.145,.698}
    \definecolor{linkcolor}{rgb}{.71,0.21,0.01}
    \definecolor{citecolor}{rgb}{.12,.54,.11}

    % ANSI colors
    \definecolor{ansi-black}{HTML}{3E424D}
    \definecolor{ansi-black-intense}{HTML}{282C36}
    \definecolor{ansi-red}{HTML}{E75C58}
    \definecolor{ansi-red-intense}{HTML}{B22B31}
    \definecolor{ansi-green}{HTML}{00A250}
    \definecolor{ansi-green-intense}{HTML}{007427}
    \definecolor{ansi-yellow}{HTML}{DDB62B}
    \definecolor{ansi-yellow-intense}{HTML}{B27D12}
    \definecolor{ansi-blue}{HTML}{208FFB}
    \definecolor{ansi-blue-intense}{HTML}{0065CA}
    \definecolor{ansi-magenta}{HTML}{D160C4}
    \definecolor{ansi-magenta-intense}{HTML}{A03196}
    \definecolor{ansi-cyan}{HTML}{60C6C8}
    \definecolor{ansi-cyan-intense}{HTML}{258F8F}
    \definecolor{ansi-white}{HTML}{C5C1B4}
    \definecolor{ansi-white-intense}{HTML}{A1A6B2}
    \definecolor{ansi-default-inverse-fg}{HTML}{FFFFFF}
    \definecolor{ansi-default-inverse-bg}{HTML}{000000}

    % commands and environments needed by pandoc snippets
    % extracted from the output of `pandoc -s`
    \providecommand{\tightlist}{%
      \setlength{\itemsep}{0pt}\setlength{\parskip}{0pt}}
    \DefineVerbatimEnvironment{Highlighting}{Verbatim}{commandchars=\\\{\}}
    % Add ',fontsize=\small' for more characters per line
    \newenvironment{Shaded}{}{}
    \newcommand{\KeywordTok}[1]{\textcolor[rgb]{0.00,0.44,0.13}{\textbf{{#1}}}}
    \newcommand{\DataTypeTok}[1]{\textcolor[rgb]{0.56,0.13,0.00}{{#1}}}
    \newcommand{\DecValTok}[1]{\textcolor[rgb]{0.25,0.63,0.44}{{#1}}}
    \newcommand{\BaseNTok}[1]{\textcolor[rgb]{0.25,0.63,0.44}{{#1}}}
    \newcommand{\FloatTok}[1]{\textcolor[rgb]{0.25,0.63,0.44}{{#1}}}
    \newcommand{\CharTok}[1]{\textcolor[rgb]{0.25,0.44,0.63}{{#1}}}
    \newcommand{\StringTok}[1]{\textcolor[rgb]{0.25,0.44,0.63}{{#1}}}
    \newcommand{\CommentTok}[1]{\textcolor[rgb]{0.38,0.63,0.69}{\textit{{#1}}}}
    \newcommand{\OtherTok}[1]{\textcolor[rgb]{0.00,0.44,0.13}{{#1}}}
    \newcommand{\AlertTok}[1]{\textcolor[rgb]{1.00,0.00,0.00}{\textbf{{#1}}}}
    \newcommand{\FunctionTok}[1]{\textcolor[rgb]{0.02,0.16,0.49}{{#1}}}
    \newcommand{\RegionMarkerTok}[1]{{#1}}
    \newcommand{\ErrorTok}[1]{\textcolor[rgb]{1.00,0.00,0.00}{\textbf{{#1}}}}
    \newcommand{\NormalTok}[1]{{#1}}
    
    % Additional commands for more recent versions of Pandoc
    \newcommand{\ConstantTok}[1]{\textcolor[rgb]{0.53,0.00,0.00}{{#1}}}
    \newcommand{\SpecialCharTok}[1]{\textcolor[rgb]{0.25,0.44,0.63}{{#1}}}
    \newcommand{\VerbatimStringTok}[1]{\textcolor[rgb]{0.25,0.44,0.63}{{#1}}}
    \newcommand{\SpecialStringTok}[1]{\textcolor[rgb]{0.73,0.40,0.53}{{#1}}}
    \newcommand{\ImportTok}[1]{{#1}}
    \newcommand{\DocumentationTok}[1]{\textcolor[rgb]{0.73,0.13,0.13}{\textit{{#1}}}}
    \newcommand{\AnnotationTok}[1]{\textcolor[rgb]{0.38,0.63,0.69}{\textbf{\textit{{#1}}}}}
    \newcommand{\CommentVarTok}[1]{\textcolor[rgb]{0.38,0.63,0.69}{\textbf{\textit{{#1}}}}}
    \newcommand{\VariableTok}[1]{\textcolor[rgb]{0.10,0.09,0.49}{{#1}}}
    \newcommand{\ControlFlowTok}[1]{\textcolor[rgb]{0.00,0.44,0.13}{\textbf{{#1}}}}
    \newcommand{\OperatorTok}[1]{\textcolor[rgb]{0.40,0.40,0.40}{{#1}}}
    \newcommand{\BuiltInTok}[1]{{#1}}
    \newcommand{\ExtensionTok}[1]{{#1}}
    \newcommand{\PreprocessorTok}[1]{\textcolor[rgb]{0.74,0.48,0.00}{{#1}}}
    \newcommand{\AttributeTok}[1]{\textcolor[rgb]{0.49,0.56,0.16}{{#1}}}
    \newcommand{\InformationTok}[1]{\textcolor[rgb]{0.38,0.63,0.69}{\textbf{\textit{{#1}}}}}
    \newcommand{\WarningTok}[1]{\textcolor[rgb]{0.38,0.63,0.69}{\textbf{\textit{{#1}}}}}
    
    
    % Define a nice break command that doesn't care if a line doesn't already
    % exist.
    \def\br{\hspace*{\fill} \\* }
    % Math Jax compatibility definitions
    \def\gt{>}
    \def\lt{<}
    \let\Oldtex\TeX
    \let\Oldlatex\LaTeX
    \renewcommand{\TeX}{\textrm{\Oldtex}}
    \renewcommand{\LaTeX}{\textrm{\Oldlatex}}
    % Document parameters
    % Document title
    \title{fwliir}
    
    
    
    
    
% Pygments definitions
\makeatletter
\def\PY@reset{\let\PY@it=\relax \let\PY@bf=\relax%
    \let\PY@ul=\relax \let\PY@tc=\relax%
    \let\PY@bc=\relax \let\PY@ff=\relax}
\def\PY@tok#1{\csname PY@tok@#1\endcsname}
\def\PY@toks#1+{\ifx\relax#1\empty\else%
    \PY@tok{#1}\expandafter\PY@toks\fi}
\def\PY@do#1{\PY@bc{\PY@tc{\PY@ul{%
    \PY@it{\PY@bf{\PY@ff{#1}}}}}}}
\def\PY#1#2{\PY@reset\PY@toks#1+\relax+\PY@do{#2}}

\expandafter\def\csname PY@tok@w\endcsname{\def\PY@tc##1{\textcolor[rgb]{0.73,0.73,0.73}{##1}}}
\expandafter\def\csname PY@tok@c\endcsname{\let\PY@it=\textit\def\PY@tc##1{\textcolor[rgb]{0.25,0.50,0.50}{##1}}}
\expandafter\def\csname PY@tok@cp\endcsname{\def\PY@tc##1{\textcolor[rgb]{0.74,0.48,0.00}{##1}}}
\expandafter\def\csname PY@tok@k\endcsname{\let\PY@bf=\textbf\def\PY@tc##1{\textcolor[rgb]{0.00,0.50,0.00}{##1}}}
\expandafter\def\csname PY@tok@kp\endcsname{\def\PY@tc##1{\textcolor[rgb]{0.00,0.50,0.00}{##1}}}
\expandafter\def\csname PY@tok@kt\endcsname{\def\PY@tc##1{\textcolor[rgb]{0.69,0.00,0.25}{##1}}}
\expandafter\def\csname PY@tok@o\endcsname{\def\PY@tc##1{\textcolor[rgb]{0.40,0.40,0.40}{##1}}}
\expandafter\def\csname PY@tok@ow\endcsname{\let\PY@bf=\textbf\def\PY@tc##1{\textcolor[rgb]{0.67,0.13,1.00}{##1}}}
\expandafter\def\csname PY@tok@nb\endcsname{\def\PY@tc##1{\textcolor[rgb]{0.00,0.50,0.00}{##1}}}
\expandafter\def\csname PY@tok@nf\endcsname{\def\PY@tc##1{\textcolor[rgb]{0.00,0.00,1.00}{##1}}}
\expandafter\def\csname PY@tok@nc\endcsname{\let\PY@bf=\textbf\def\PY@tc##1{\textcolor[rgb]{0.00,0.00,1.00}{##1}}}
\expandafter\def\csname PY@tok@nn\endcsname{\let\PY@bf=\textbf\def\PY@tc##1{\textcolor[rgb]{0.00,0.00,1.00}{##1}}}
\expandafter\def\csname PY@tok@ne\endcsname{\let\PY@bf=\textbf\def\PY@tc##1{\textcolor[rgb]{0.82,0.25,0.23}{##1}}}
\expandafter\def\csname PY@tok@nv\endcsname{\def\PY@tc##1{\textcolor[rgb]{0.10,0.09,0.49}{##1}}}
\expandafter\def\csname PY@tok@no\endcsname{\def\PY@tc##1{\textcolor[rgb]{0.53,0.00,0.00}{##1}}}
\expandafter\def\csname PY@tok@nl\endcsname{\def\PY@tc##1{\textcolor[rgb]{0.63,0.63,0.00}{##1}}}
\expandafter\def\csname PY@tok@ni\endcsname{\let\PY@bf=\textbf\def\PY@tc##1{\textcolor[rgb]{0.60,0.60,0.60}{##1}}}
\expandafter\def\csname PY@tok@na\endcsname{\def\PY@tc##1{\textcolor[rgb]{0.49,0.56,0.16}{##1}}}
\expandafter\def\csname PY@tok@nt\endcsname{\let\PY@bf=\textbf\def\PY@tc##1{\textcolor[rgb]{0.00,0.50,0.00}{##1}}}
\expandafter\def\csname PY@tok@nd\endcsname{\def\PY@tc##1{\textcolor[rgb]{0.67,0.13,1.00}{##1}}}
\expandafter\def\csname PY@tok@s\endcsname{\def\PY@tc##1{\textcolor[rgb]{0.73,0.13,0.13}{##1}}}
\expandafter\def\csname PY@tok@sd\endcsname{\let\PY@it=\textit\def\PY@tc##1{\textcolor[rgb]{0.73,0.13,0.13}{##1}}}
\expandafter\def\csname PY@tok@si\endcsname{\let\PY@bf=\textbf\def\PY@tc##1{\textcolor[rgb]{0.73,0.40,0.53}{##1}}}
\expandafter\def\csname PY@tok@se\endcsname{\let\PY@bf=\textbf\def\PY@tc##1{\textcolor[rgb]{0.73,0.40,0.13}{##1}}}
\expandafter\def\csname PY@tok@sr\endcsname{\def\PY@tc##1{\textcolor[rgb]{0.73,0.40,0.53}{##1}}}
\expandafter\def\csname PY@tok@ss\endcsname{\def\PY@tc##1{\textcolor[rgb]{0.10,0.09,0.49}{##1}}}
\expandafter\def\csname PY@tok@sx\endcsname{\def\PY@tc##1{\textcolor[rgb]{0.00,0.50,0.00}{##1}}}
\expandafter\def\csname PY@tok@m\endcsname{\def\PY@tc##1{\textcolor[rgb]{0.40,0.40,0.40}{##1}}}
\expandafter\def\csname PY@tok@gh\endcsname{\let\PY@bf=\textbf\def\PY@tc##1{\textcolor[rgb]{0.00,0.00,0.50}{##1}}}
\expandafter\def\csname PY@tok@gu\endcsname{\let\PY@bf=\textbf\def\PY@tc##1{\textcolor[rgb]{0.50,0.00,0.50}{##1}}}
\expandafter\def\csname PY@tok@gd\endcsname{\def\PY@tc##1{\textcolor[rgb]{0.63,0.00,0.00}{##1}}}
\expandafter\def\csname PY@tok@gi\endcsname{\def\PY@tc##1{\textcolor[rgb]{0.00,0.63,0.00}{##1}}}
\expandafter\def\csname PY@tok@gr\endcsname{\def\PY@tc##1{\textcolor[rgb]{1.00,0.00,0.00}{##1}}}
\expandafter\def\csname PY@tok@ge\endcsname{\let\PY@it=\textit}
\expandafter\def\csname PY@tok@gs\endcsname{\let\PY@bf=\textbf}
\expandafter\def\csname PY@tok@gp\endcsname{\let\PY@bf=\textbf\def\PY@tc##1{\textcolor[rgb]{0.00,0.00,0.50}{##1}}}
\expandafter\def\csname PY@tok@go\endcsname{\def\PY@tc##1{\textcolor[rgb]{0.53,0.53,0.53}{##1}}}
\expandafter\def\csname PY@tok@gt\endcsname{\def\PY@tc##1{\textcolor[rgb]{0.00,0.27,0.87}{##1}}}
\expandafter\def\csname PY@tok@err\endcsname{\def\PY@bc##1{\setlength{\fboxsep}{0pt}\fcolorbox[rgb]{1.00,0.00,0.00}{1,1,1}{\strut ##1}}}
\expandafter\def\csname PY@tok@kc\endcsname{\let\PY@bf=\textbf\def\PY@tc##1{\textcolor[rgb]{0.00,0.50,0.00}{##1}}}
\expandafter\def\csname PY@tok@kd\endcsname{\let\PY@bf=\textbf\def\PY@tc##1{\textcolor[rgb]{0.00,0.50,0.00}{##1}}}
\expandafter\def\csname PY@tok@kn\endcsname{\let\PY@bf=\textbf\def\PY@tc##1{\textcolor[rgb]{0.00,0.50,0.00}{##1}}}
\expandafter\def\csname PY@tok@kr\endcsname{\let\PY@bf=\textbf\def\PY@tc##1{\textcolor[rgb]{0.00,0.50,0.00}{##1}}}
\expandafter\def\csname PY@tok@bp\endcsname{\def\PY@tc##1{\textcolor[rgb]{0.00,0.50,0.00}{##1}}}
\expandafter\def\csname PY@tok@fm\endcsname{\def\PY@tc##1{\textcolor[rgb]{0.00,0.00,1.00}{##1}}}
\expandafter\def\csname PY@tok@vc\endcsname{\def\PY@tc##1{\textcolor[rgb]{0.10,0.09,0.49}{##1}}}
\expandafter\def\csname PY@tok@vg\endcsname{\def\PY@tc##1{\textcolor[rgb]{0.10,0.09,0.49}{##1}}}
\expandafter\def\csname PY@tok@vi\endcsname{\def\PY@tc##1{\textcolor[rgb]{0.10,0.09,0.49}{##1}}}
\expandafter\def\csname PY@tok@vm\endcsname{\def\PY@tc##1{\textcolor[rgb]{0.10,0.09,0.49}{##1}}}
\expandafter\def\csname PY@tok@sa\endcsname{\def\PY@tc##1{\textcolor[rgb]{0.73,0.13,0.13}{##1}}}
\expandafter\def\csname PY@tok@sb\endcsname{\def\PY@tc##1{\textcolor[rgb]{0.73,0.13,0.13}{##1}}}
\expandafter\def\csname PY@tok@sc\endcsname{\def\PY@tc##1{\textcolor[rgb]{0.73,0.13,0.13}{##1}}}
\expandafter\def\csname PY@tok@dl\endcsname{\def\PY@tc##1{\textcolor[rgb]{0.73,0.13,0.13}{##1}}}
\expandafter\def\csname PY@tok@s2\endcsname{\def\PY@tc##1{\textcolor[rgb]{0.73,0.13,0.13}{##1}}}
\expandafter\def\csname PY@tok@sh\endcsname{\def\PY@tc##1{\textcolor[rgb]{0.73,0.13,0.13}{##1}}}
\expandafter\def\csname PY@tok@s1\endcsname{\def\PY@tc##1{\textcolor[rgb]{0.73,0.13,0.13}{##1}}}
\expandafter\def\csname PY@tok@mb\endcsname{\def\PY@tc##1{\textcolor[rgb]{0.40,0.40,0.40}{##1}}}
\expandafter\def\csname PY@tok@mf\endcsname{\def\PY@tc##1{\textcolor[rgb]{0.40,0.40,0.40}{##1}}}
\expandafter\def\csname PY@tok@mh\endcsname{\def\PY@tc##1{\textcolor[rgb]{0.40,0.40,0.40}{##1}}}
\expandafter\def\csname PY@tok@mi\endcsname{\def\PY@tc##1{\textcolor[rgb]{0.40,0.40,0.40}{##1}}}
\expandafter\def\csname PY@tok@il\endcsname{\def\PY@tc##1{\textcolor[rgb]{0.40,0.40,0.40}{##1}}}
\expandafter\def\csname PY@tok@mo\endcsname{\def\PY@tc##1{\textcolor[rgb]{0.40,0.40,0.40}{##1}}}
\expandafter\def\csname PY@tok@ch\endcsname{\let\PY@it=\textit\def\PY@tc##1{\textcolor[rgb]{0.25,0.50,0.50}{##1}}}
\expandafter\def\csname PY@tok@cm\endcsname{\let\PY@it=\textit\def\PY@tc##1{\textcolor[rgb]{0.25,0.50,0.50}{##1}}}
\expandafter\def\csname PY@tok@cpf\endcsname{\let\PY@it=\textit\def\PY@tc##1{\textcolor[rgb]{0.25,0.50,0.50}{##1}}}
\expandafter\def\csname PY@tok@c1\endcsname{\let\PY@it=\textit\def\PY@tc##1{\textcolor[rgb]{0.25,0.50,0.50}{##1}}}
\expandafter\def\csname PY@tok@cs\endcsname{\let\PY@it=\textit\def\PY@tc##1{\textcolor[rgb]{0.25,0.50,0.50}{##1}}}

\def\PYZbs{\char`\\}
\def\PYZus{\char`\_}
\def\PYZob{\char`\{}
\def\PYZcb{\char`\}}
\def\PYZca{\char`\^}
\def\PYZam{\char`\&}
\def\PYZlt{\char`\<}
\def\PYZgt{\char`\>}
\def\PYZsh{\char`\#}
\def\PYZpc{\char`\%}
\def\PYZdl{\char`\$}
\def\PYZhy{\char`\-}
\def\PYZsq{\char`\'}
\def\PYZdq{\char`\"}
\def\PYZti{\char`\~}
% for compatibility with earlier versions
\def\PYZat{@}
\def\PYZlb{[}
\def\PYZrb{]}
\makeatother


    % For linebreaks inside Verbatim environment from package fancyvrb. 
    \makeatletter
        \newbox\Wrappedcontinuationbox 
        \newbox\Wrappedvisiblespacebox 
        \newcommand*\Wrappedvisiblespace {\textcolor{red}{\textvisiblespace}} 
        \newcommand*\Wrappedcontinuationsymbol {\textcolor{red}{\llap{\tiny$\m@th\hookrightarrow$}}} 
        \newcommand*\Wrappedcontinuationindent {3ex } 
        \newcommand*\Wrappedafterbreak {\kern\Wrappedcontinuationindent\copy\Wrappedcontinuationbox} 
        % Take advantage of the already applied Pygments mark-up to insert 
        % potential linebreaks for TeX processing. 
        %        {, <, #, %, $, ' and ": go to next line. 
        %        _, }, ^, &, >, - and ~: stay at end of broken line. 
        % Use of \textquotesingle for straight quote. 
        \newcommand*\Wrappedbreaksatspecials {% 
            \def\PYGZus{\discretionary{\char`\_}{\Wrappedafterbreak}{\char`\_}}% 
            \def\PYGZob{\discretionary{}{\Wrappedafterbreak\char`\{}{\char`\{}}% 
            \def\PYGZcb{\discretionary{\char`\}}{\Wrappedafterbreak}{\char`\}}}% 
            \def\PYGZca{\discretionary{\char`\^}{\Wrappedafterbreak}{\char`\^}}% 
            \def\PYGZam{\discretionary{\char`\&}{\Wrappedafterbreak}{\char`\&}}% 
            \def\PYGZlt{\discretionary{}{\Wrappedafterbreak\char`\<}{\char`\<}}% 
            \def\PYGZgt{\discretionary{\char`\>}{\Wrappedafterbreak}{\char`\>}}% 
            \def\PYGZsh{\discretionary{}{\Wrappedafterbreak\char`\#}{\char`\#}}% 
            \def\PYGZpc{\discretionary{}{\Wrappedafterbreak\char`\%}{\char`\%}}% 
            \def\PYGZdl{\discretionary{}{\Wrappedafterbreak\char`\$}{\char`\$}}% 
            \def\PYGZhy{\discretionary{\char`\-}{\Wrappedafterbreak}{\char`\-}}% 
            \def\PYGZsq{\discretionary{}{\Wrappedafterbreak\textquotesingle}{\textquotesingle}}% 
            \def\PYGZdq{\discretionary{}{\Wrappedafterbreak\char`\"}{\char`\"}}% 
            \def\PYGZti{\discretionary{\char`\~}{\Wrappedafterbreak}{\char`\~}}% 
        } 
        % Some characters . , ; ? ! / are not pygmentized. 
        % This macro makes them "active" and they will insert potential linebreaks 
        \newcommand*\Wrappedbreaksatpunct {% 
            \lccode`\~`\.\lowercase{\def~}{\discretionary{\hbox{\char`\.}}{\Wrappedafterbreak}{\hbox{\char`\.}}}% 
            \lccode`\~`\,\lowercase{\def~}{\discretionary{\hbox{\char`\,}}{\Wrappedafterbreak}{\hbox{\char`\,}}}% 
            \lccode`\~`\;\lowercase{\def~}{\discretionary{\hbox{\char`\;}}{\Wrappedafterbreak}{\hbox{\char`\;}}}% 
            \lccode`\~`\:\lowercase{\def~}{\discretionary{\hbox{\char`\:}}{\Wrappedafterbreak}{\hbox{\char`\:}}}% 
            \lccode`\~`\?\lowercase{\def~}{\discretionary{\hbox{\char`\?}}{\Wrappedafterbreak}{\hbox{\char`\?}}}% 
            \lccode`\~`\!\lowercase{\def~}{\discretionary{\hbox{\char`\!}}{\Wrappedafterbreak}{\hbox{\char`\!}}}% 
            \lccode`\~`\/\lowercase{\def~}{\discretionary{\hbox{\char`\/}}{\Wrappedafterbreak}{\hbox{\char`\/}}}% 
            \catcode`\.\active
            \catcode`\,\active 
            \catcode`\;\active
            \catcode`\:\active
            \catcode`\?\active
            \catcode`\!\active
            \catcode`\/\active 
            \lccode`\~`\~ 	
        }
    \makeatother

    \let\OriginalVerbatim=\Verbatim
    \makeatletter
    \renewcommand{\Verbatim}[1][1]{%
        %\parskip\z@skip
        \sbox\Wrappedcontinuationbox {\Wrappedcontinuationsymbol}%
        \sbox\Wrappedvisiblespacebox {\FV@SetupFont\Wrappedvisiblespace}%
        \def\FancyVerbFormatLine ##1{\hsize\linewidth
            \vtop{\raggedright\hyphenpenalty\z@\exhyphenpenalty\z@
                \doublehyphendemerits\z@\finalhyphendemerits\z@
                \strut ##1\strut}%
        }%
        % If the linebreak is at a space, the latter will be displayed as visible
        % space at end of first line, and a continuation symbol starts next line.
        % Stretch/shrink are however usually zero for typewriter font.
        \def\FV@Space {%
            \nobreak\hskip\z@ plus\fontdimen3\font minus\fontdimen4\font
            \discretionary{\copy\Wrappedvisiblespacebox}{\Wrappedafterbreak}
            {\kern\fontdimen2\font}%
        }%
        
        % Allow breaks at special characters using \PYG... macros.
        \Wrappedbreaksatspecials
        % Breaks at punctuation characters . , ; ? ! and / need catcode=\active 	
        \OriginalVerbatim[#1,codes*=\Wrappedbreaksatpunct]%
    }
    \makeatother

    % Exact colors from NB
    \definecolor{incolor}{HTML}{303F9F}
    \definecolor{outcolor}{HTML}{D84315}
    \definecolor{cellborder}{HTML}{CFCFCF}
    \definecolor{cellbackground}{HTML}{F7F7F7}
    
    % prompt
    \newcommand{\prompt}[4]{
        \llap{{\color{#2}[#3]: #4}}\vspace{-1.25em}
    }
    

    
    % Prevent overflowing lines due to hard-to-break entities
    \sloppy 
    % Setup hyperref package
    \hypersetup{
      breaklinks=true,  % so long urls are correctly broken across lines
      colorlinks=true,
      urlcolor=urlcolor,
      linkcolor=linkcolor,
      citecolor=citecolor,
      }
    % Slightly bigger margins than the latex defaults
    
    \geometry{verbose,tmargin=1in,bmargin=1in,lmargin=1in,rmargin=1in}
    
    

    \begin{document}
    
    \section{Aplicación de algoritmos genéticos al diseño de filtros IIR en
punto
fijo}\label{aplicaciuxf3n-de-algoritmos-genuxe9ticos-al-diseuxf1o-de-filtros-iir-en-punto-fijo}

    \subsection{Extracto}\label{extracto}

    El desempeño de los filtros digitales en general, y de los filtros de
respuesta impulsional infinita o IIR en particular¹, se ve afectado por
las representaciones numéricas finitas utilizadas para su
implementación. Las herramientas de análisis y síntesis típicas²,
lineales, no intentan modelar los efectos que impone una aritmética con
restricciones³. Ésto se torna particularmente notorio en
representaciones de punto fijo de relativo bajo número de bits⁴
-valiosas por su sencillez, reducido costo y alta velocidad-, al punto
de deformar completamente la respuesta del filtro proyectado.

Se vuelve necesario recurrir a otros métodos para arribar a una solución
satisfactoria, maximizando una métrica cuantificable de ese grado de
satisfacción, en ocasiones en detrimento de otros aspectos de la
solución. Los algoritmos genéticos, una subclase dentro del conjunto de
algoritmos evolutivos, llevan a cabo esta búsqueda mediante operaciones
inspiradas en el proceso de selección natural. De carácter
probabilístico, no garantizan la solución óptima pero permiten la
exploración de universos de discurso extensos en un tiempo acotado.

¹ Por caso, la realimentación combinada con operatorias de truncado y
desborde puede resultar en oscilasciones sostenidas.

² Por caso, la transformada Z y el mapeo bilineal en tándem con
\emph{frequency warping} para extender soluciones en tiempo continuo a
tiempo discreto.

³ Producto de acumuladores finitos para albergar los resultados
parciales de sumas y multiplicaciones, el redondeo, truncado y/o
saturación necesarios para mantener el resultado en valores
representables, etc.

⁴ Conforme el número de bits crece y se abandona el punto fijo por el
punto flotante, la discrepancia entre representación continua y discreta
se vuelve despreciable.

    \subsection{Estudio preliminar}\label{estudio-preliminar}

    \subsubsection{Procesamiento digital de
señales}\label{procesamiento-digital-de-seuxf1ales}

\paragraph{Filtrado en tiempo
discreto}\label{filtrado-en-tiempo-discreto}

Todo filtro puede ser descripto por su ecuación en diferencias:


\begin{equation}
y[k] = \sum^N_{n = 0} b_n \cdot x[k - n] - \sum^M_{m = 1} a_m \cdot y[k - m]
\end{equation}

Esta ecuación permite computar la respuesta muestra a muestra (respuesta
temporal si se asume un período de muestreo \(t_s\) dado) e indica la
estructura más simple (directa) del filtro.

    \begin{center}
    \adjustimage{max size={0.9\linewidth}{0.9\paperheight}}{output_6_0.png}
    \end{center}
    { \hspace*{\fill} \\}
    

    Asumiendo coeficientes \(b_n\) y \(a_m\) invariantes en el tiempo, la
transformada Z de la ecuación en diferencias resulta:


\begin{equation}
\frac{Y(z)}{X(z)} = \frac{\sum^N_{n = 0} b_n \cdot z^{-n}}{1 + \sum^M_{m = 1} a_m \cdot z^{-m}} \\ 
z = re^{j\Omega}
\end{equation}

Las raíces del polinomio numerador de coeficientes \(b_n\) constituyen
los denominados ceros \(z_n\) de esta función transferencia y las raíces
del polinomio denominador de coeficientes \(a_m\) constituyen los polos
\(p_m\). Para \(\|z\| = r = 1\), obtenemos la respuesta en frecuencia
del filtro (esto es, la transformada de Fourier en tiempo discreto).

    \paragraph{Efectos de la representación
numérica}\label{efectos-de-la-representaciuxf3n-numuxe9rica}

Los valores de entrada \(x[k]\), de salida \(y[k]\), y coeficientes
\(b_n\) y \(a_m\) deben ser representados en forma binaria para su
procesamiento. En lo que sigue, se asume una representación binaria
signada en complemento a 2 de punto fijo de \(N_e.N_f\) bits, que
permite procesar magnitudes fraccionales utilizando unidades
aritmético-lógicas para enteros signados:

\[ -2^{N_b-1} \leq x \leq 2^{N_b-1}-\frac{1}{2^{N_f}} \]

Coeficientes producto un diseño con mayor precisión numérica deben ser
truncados o redondeados para ser representados de esta forma. El
resultado parcial de cada multiplicación requiere \(2(N_b + N_f)\) bits
para su representación exacta, lo cual también implica, tarde o
temprano, su truncado o redondeo. Las sucesivas sumas deben alojarse en
un acumulador de \(N_a\) bits, por lo que están sujetas a desborde o
saturación.

Cada una de estas operatorias no lineales reduce el rango dinámico
efectivo de la representación y distorsiona el filtro predeterminado por
otros medios.

    \subsubsection{Algoritmo genético}\label{algoritmo-genuxe9tico}

En su forma más simple, un algoritmo de esta clase puede ser descripto
por el siguiente diagrama de flujo:

    
    \begin{center}
    \adjustimage{max size={0.9\linewidth}{0.75\paperheight}}{output_10_0.png}
    \end{center}
    { \hspace*{\fill} \\}
    

    Una población inicial de individuos, en principio generada de forma
aleatoria -pero bien puede ser el resultado de conocimiento o procesos
previos-, es seleccionada en función de la aptitud de su fenotipo (o
conjunto de características observables) y luego sometida a operadores
de cruza y mutación de su genotipo (o conjunto de genes que codifica
dichas características) para generar descendencia. Esta operatoria se
repite a lo largo de generaciones hasta que una condición de fin o
término dada se cumple. Los genes que redundan en individuos más aptos
tenderán, en promedio, a incrementar su frecuencia de ocurrencia.

También en su forma más simple, la implementación de este algoritmo
codifica los genes en secuencias binarias y opera sobre subconjuntos de
éstas. Como se verá más adelante, el desempeño de esta heurística
inspirada en la biología es fuertemente dependiente de la representación
del genotipo de los individuos, de la diversidad genética inicial, de la
clausura (o su ausencia) de los operadores genéticos respecto al
universo de discurso y de la tendencia de los mismos a preservar los
genes más aptos a través de las generaciones, etc., al punto de exhibir
severos problemas de convergencia si el delicado balance entre los
factores involucrados se ve afectado.

    \subsection{Aplicación}\label{aplicaciuxf3n}

    \subsubsection{Algoritmo}\label{algoritmo}

\paragraph{Genotipo (o
representación)}\label{genotipo-o-representaciuxf3n}

Sin hipótesis simplificadora alguna, una representación compacta de un
filtro digital requiere de todos sus coeficientes.

Parte de las dificultades en la síntesis de filtros digitales en punto
fijo radica en el hecho de que el procedimiento normalmente se efectúa
en \(\mathbb{R}\), en lugar del subconjunto \(U \subset \mathbb{R}\) con
aritmética modular en el que se lo implementa. Por ello, los
coeficientes se representarán en \(1.N_b\) desde un comienzo.

La estructura del filtro digital también tiene consecuencias sobre su
desempeño. La sensibilidad a los errores numéricos crece con el orden.
Las sumas parciales pueden ocasionar el desborde o saturación de los
acumuladores que las almacenan. La decomposición en secciones
bicuadráticas, de segundo orden, y su ordenamiento tal que se minimize
la posibilidad de desborde o saturación es una solución tipica en
implementaciones tradicionales a estos problemas.

Con estas consideraciones, el genotipo de cada filtro queda conformado
por una cascada de secciones bicuadráticas
\([SOS_0, SOS_1, ..., SOS_i]\), donde la ecuación en diferencias de la
sección \(SOS_i\) es:


\begin{equation}
y_i[k] = c_0 \cdot \Bigl( b_0 \cdot x_i[k] + b_1 \cdot x_i[k-1] + b_2 \cdot x_i[k-2] - a_1 \cdot y_i[k-1] - a_2 \cdot y_i[k-2] \Bigr)
\end{equation}


o simplemente \([b_0, b_1, b_2, a_1, a_2, c_0]\).

    
    \begin{center}
    \adjustimage{max size={0.9\linewidth}{0.9\paperheight}}{output_14_0.png}
    \end{center}
    { \hspace*{\fill} \\}
    

    Notar que el término \(a_0\) queda tácito, pues es siempre 1. Para
lidiar con órdenes impares, una sección de primer orden puede
introducirse haciendo \(b_2 = a_2 = 0\). El agregado de una ganancia de
etapa \(c_0\) permite cambios de escala de los coeficientes a modo de
mantenerlos en el intervalo de valores representables.

\paragraph{Operadores}\label{operadores}

\subparagraph{Evaluación}\label{evaluaciuxf3n}

La aptitud de cada filtro se evalúa en relación al desempeño deseado. En
la síntesis tradicional, el filtro exhibe un respuesta al impulso o una
respuesta en frecuencia dadas o con ciertas características a partir de
la utilización de teoría de aproximación, muestreo, convolución con
ventanas, etc. En este caso, de forma empírica se ha determinado que la
métrica de aptitud que mejor resultados arroja es:

\[
f_k = \frac{1}{\mu_{er_k}^2 + \sigma_{er_k}^2}
\]

donde

\[
er_k[i] = \frac{h_k[i] - h_t[i]}{\underset{i}{max} \|h_t[i]\|}
\]

es el error relativo de la respuesta impulsional del filtro \(h_k\)
respecto a la respuesta impulsional de la plantilla \(h_t\);

\[
\mu_{er_k} = \frac{1}{N} \sum^N_{i = 0} er_k[i]
\]

es la media de dicho error relativo \(er_k\);

\[
\sigma^2_{er_k} = \frac{1}{N - 1} \sum^N_{i = 0} (er_k[i] - \mu_{er_k})^2
\]

es la varianza de dicho error relativo \(er_k\). El valor apropiado de N
dependerá de cada respuesta a aproximar.

Observar que el denominador de la función \(f_k\) es numéricamente
equivalente a la potencia de \(er_k\) entendida como variable aleatoria.

Es importante mencionar, sin embargo, que el cómputo de la respuesta
impulsional \(h_k\) durante y sólo durante la evaluación del filtro se
realiza en su representación normalizada con la máxima precisión
numérica disponible. El uso de aritmética en punto fijo durante el
proceso evolutivo degrada severamente la convergencia.

\subparagraph{Selección}\label{selecciuxf3n}

Los operadores de selección mediante muestreo de la población, tomando
la aptitud, o bien una estratificación construida a partir de ella, como
probabilidad de ocurrencia no ha mostrado buen desempeño.
Frecuentemente, el algoritmo presentaba problemas de convergencia.

El operador de torneo, o la selección determínistica del más apto en
grupos conformados de forma aleatoria, ha mostrado mejores resultado y
es la que esta instancia del algoritmo genético emplea. Es, además,
fácilmente ajustable: la presión de selección aumenta con el tamaño del
torneo. Torneos grandes atentan contra la diversidad del pool genético y
deben evitarse.

\subparagraph{Cruza}\label{cruza}

El operador de cruza debe, en promedio, favorecer la reproducción de los
genes más aptos.

Los operadores de cruza binaria, como ser el de corte en uno o dos
puntos aplicados sobre la secuencia completa de coeficientes, no ha
mostrado buen desempeño, probablemente atribuible a la elevada
variabilidad que introducen en el punto de corte (considerando el peso
desigual de cada bit en una representación numérica binaria) y sólo en
el punto de corte.

La cruza uniforme de secciones de segundo orden, es decir el intercambio
probable de a pares ordenados de secciones de los progenitores para dar
origen a su descendencia, ha mostrado mejores resultados. Más aún, la
cruza uniforme de numeradores y denominadores de la función
transferencia de cada sección, tomadas de a pares aleatorios ha mostrado
aún mejores resultados y es la que esta instancia del algoritmo genético
emplea. Se observa que este operador favorece el intercambio de polos y
ceros y su redistribución.

\subparagraph{Mutación}\label{mutaciuxf3n}

El operador de mutación debe introducir, en promedio, ligeras
perturbaciones en los genes que eviten el estancamiento en máximos
locales y contribuyan a la exploración del espacio de soluciones.

Los operadores de mutación binaria, como ser la inversión de bits, no
han mostrado buen desempeño, probablemente atribuible al peso desigual
de cada bit en una representación numérica binaria y el consiguiente
efecto en la magnitud de las perturbaciones (notar el impacto de la
inversión de bit menos significativo comparado con el impacto de la
inversión del bit más significativo o incluso el de signo).

La adición de una perturbación como variable aleatoria de distribución
normal a los coeficientes, entendidos como valores numéricos, ha
mostrado en cambio buenos resultados.

\paragraph{Variaciones}\label{variaciones}

\subparagraph{Población inicial}\label{poblaciuxf3n-inicial}

La generación de la población inicial, si bien aleatoria, se limita a
conformar filtros con secciones de segundo orden estables y de fase
mínima (es decir, polos y ceros dentro o sobre la circunferencia unidad
en el plano Z), asumiendo que la solución se hallará en la
inmediaciones. Para ello, para cada sección de segundo orden:


\begin{equation}
H(z) = \frac{b(z)}{a_0 + a_1 z^{-1} + a_2 z^{-2}}
\end{equation}


y su recíproco, se aplica el criterio de estabilidad de Jury-Marden, que
garantiza que dicha función transferencia es estable si:


\begin{equation}
a_0 = 1 \\
\|a_2\| < 1 \\
a_2 > a_1 - 1, a_1 \geq 0 \\
a_2 > -a_1 - 1, a_1 < 0
\end{equation}


Adicionalmente, y pese a que el algoritmo no intenta minimizar orden,
filtros con distinto número de secciones son gestados y sometidos al
proceso evolutivo para luego optar por los más simples.

\subparagraph{Historial de más aptos}\label{historial-de-muxe1s-aptos}

Los más individuos más aptos a lo largo de todas las generaciones son
conservados, evitando perder buenas soluciones producto del caracter
aleatorio del proceso evolutivo.

\subparagraph{Propagación elitista}\label{propagaciuxf3n-elitista}

Para mejorar la convergencia, el recambio generacional de la población
se ha modificado, introduciendo una proporción de individuos de la
población, los más aptos (o la \emph{elite}), en su descendencia y
seleccionando a los mejores del conjunto unión para la próxima
generación.

Esta proporción y el tamaño de los grupos de selección por torneo están
fuertemente ligados. Torneos grandes con una proporción de individuos de
elite considerable provocan pérdida de diversidad y convergencia
prematura.

\subparagraph{Condición de término}\label{condiciuxf3n-de-tuxe9rmino}

El algoritmo avanza la población a lo largo de un número fijo de
generaciones.

\pagebreak

\paragraph{Diagrama}\label{diagrama}

    
    \begin{center}
    \adjustimage{max size={0.9\linewidth}{0.75\paperheight}}{output_16_0.png}
    \end{center}
    { \hspace*{\fill} \\}
    

    \subsubsection{Implementación}\label{implementaciuxf3n}

El algoritmo se implementó como módulo en Python 3.6, utilizando los
paquetes \href{https://www.numpy.org/}{numpy},
\href{https://scipy.org/}{scipy} y
\href{https://deap.readthedocs.io/en/master/}{deap}.

Se reproduce debajo:

\hrulefill

    \begin{Shaded}
\begin{Highlighting}[]
\ImportTok{from}\NormalTok{ deap }\ImportTok{import}\NormalTok{ base}
\ImportTok{from}\NormalTok{ deap }\ImportTok{import}\NormalTok{ algorithms}
\ImportTok{from}\NormalTok{ deap }\ImportTok{import}\NormalTok{ tools}
\ImportTok{from}\NormalTok{ deap }\ImportTok{import}\NormalTok{ creator}

\ImportTok{import}\NormalTok{ random}
\ImportTok{import}\NormalTok{ functools}
\ImportTok{import}\NormalTok{ numpy }\ImportTok{as}\NormalTok{ np}
\ImportTok{import}\NormalTok{ scipy.signal }\ImportTok{as}\NormalTok{ signal}

\NormalTok{creator.create(}
    \StringTok{'ResponseMismatch'}\NormalTok{, base.Fitness, weights}\OperatorTok{=}\NormalTok{(}\FloatTok{1.0}\NormalTok{,)}
\NormalTok{)}
\NormalTok{creator.create(}
    \StringTok{'IIR'}\NormalTok{, }\BuiltInTok{list}\NormalTok{, fitness}\OperatorTok{=}\NormalTok{creator.ResponseMismatch, nbits}\OperatorTok{=}\BuiltInTok{int}
\NormalTok{)}
\NormalTok{IIR }\OperatorTok{=}\NormalTok{ creator.IIR}


\KeywordTok{def}\NormalTok{ fitsos(sos, nbits):}
    \CommentTok{"""}
\CommentTok{    Ajusta la ganancia de una etapa de segundo orden en punto fijo}
\CommentTok{    para que sus coeficientes puedan representarse en 1.(`nbits`-1)}

\CommentTok{    :param sos: etapa de segundo orden de filtro digital IIR}
\CommentTok{       en punto fijo.}
\CommentTok{    :param nbits: cantidad de bits de la representación en}
\CommentTok{       punto fijo de los coeficientes de la etapa.}
\CommentTok{    """}
    \CommentTok{# Computa el límite númerico de la representación entera signada.}
\NormalTok{    n }\OperatorTok{=} \DecValTok{2}\OperatorTok{**}\NormalTok{(nbits }\OperatorTok{-} \DecValTok{1}\NormalTok{)}
    \CommentTok{# Busca el coeficiente de máximo valor absoluto.}
\NormalTok{    c }\OperatorTok{=}\NormalTok{ sos[np.argmax(np.}\BuiltInTok{abs}\NormalTok{(sos[}\OperatorTok{-}\DecValTok{1}\NormalTok{] }\OperatorTok{*}\NormalTok{ sos[:}\OperatorTok{-}\DecValTok{1}\NormalTok{]))]}
    \ControlFlowTok{if} \BuiltInTok{abs}\NormalTok{(c) }\OperatorTok{>=}\NormalTok{ n:}
        \CommentTok{# Escala todos los coeficientes para que el coeficiente}
        \CommentTok{# de máximo valor absoluto sea -1 en punto fijo.}
\NormalTok{        sos[:}\OperatorTok{-}\DecValTok{1}\NormalTok{] }\OperatorTok{=} \OperatorTok{-}\NormalTok{n }\OperatorTok{*}\NormalTok{ sos[}\OperatorTok{-}\DecValTok{1}\NormalTok{] }\OperatorTok{*}\NormalTok{ sos[:}\OperatorTok{-}\DecValTok{1}\NormalTok{] }\OperatorTok{//} \BuiltInTok{abs}\NormalTok{(c)}
        \CommentTok{# Conserva el factor de escala aplicado.}
\NormalTok{        sos[}\OperatorTok{-}\DecValTok{1}\NormalTok{] }\OperatorTok{=} \BuiltInTok{min}\NormalTok{(}\OperatorTok{-}\DecValTok{1}\NormalTok{, }\OperatorTok{-}\BuiltInTok{abs}\NormalTok{(c) }\OperatorTok{//}\NormalTok{ n)}
    \ControlFlowTok{return}\NormalTok{ sos}


\KeywordTok{def}\NormalTok{ impulse(iir, ts, n):}
    \CommentTok{"""}
\CommentTok{    Computa la respuesta al impulso de un filtro digital IIR}
\CommentTok{    en punto fijo.}

\CommentTok{    El filtro se representa como una secuencia ordenada de}
\CommentTok{    etapas de segundo orden, cada una representada por los}
\CommentTok{    coeficientes de su ecuación en diferencias más una}
\CommentTok{    ganancia: [b0, b1, b2, a1, a2, k]. El coeficiente a0 se}
\CommentTok{    asume unitario. Etapas de primer orden pueden obtenerse}
\CommentTok{    haciendo b2 = a2 = 0.}

\CommentTok{    :param iir: filtro digital IIR en punto fijo.}
\CommentTok{    :param ts: período de muestreo, en segundos.}
\CommentTok{    :param n: cantidad de muestras.}
\CommentTok{    :return: tiempo y respuesta al impulso, como tupla.}
\CommentTok{    """}
    \CommentTok{# Inicializa en cero los vectores de salida de}
    \CommentTok{# cada una de las M etapas del filtro más el de}
    \CommentTok{# entrada. Cada vector puede contener N muestras}
    \CommentTok{# más 2 muestras adicionales para conformar la}
    \CommentTok{# línea de demora.}
\NormalTok{    y }\OperatorTok{=}\NormalTok{ np.zeros([}\BuiltInTok{len}\NormalTok{(iir) }\OperatorTok{+} \DecValTok{1}\NormalTok{, n }\OperatorTok{+} \DecValTok{2}\NormalTok{], dtype}\OperatorTok{=}\BuiltInTok{int}\NormalTok{)}
    \CommentTok{# Inicializar el vector de entrada con un delta}
    \CommentTok{# discreto.}
\NormalTok{    y[}\DecValTok{0}\NormalTok{, }\DecValTok{2}\NormalTok{] }\OperatorTok{=} \DecValTok{2}\OperatorTok{**}\NormalTok{(iir.nbits}\DecValTok{-1}\NormalTok{) }\OperatorTok{-} \DecValTok{1}
    \CommentTok{# Computar para cada instante discreto las salidas}
    \CommentTok{# de cada etapa, desde la primera hasta la última}
    \CommentTok{# en ese orden.}
    \ControlFlowTok{for}\NormalTok{ j }\KeywordTok{in} \BuiltInTok{range}\NormalTok{(}\DecValTok{2}\NormalTok{, n }\OperatorTok{+} \DecValTok{2}\NormalTok{):}
        \ControlFlowTok{for}\NormalTok{ i, sos }\KeywordTok{in} \BuiltInTok{enumerate}\NormalTok{(iir, start}\OperatorTok{=}\DecValTok{1}\NormalTok{):}
\NormalTok{            b0, b1, b2, a1, a2, k }\OperatorTok{=}\NormalTok{ sos}
            \CommentTok{# Computar la ecuación de diferencias, truncando}
            \CommentTok{# y saturando el resultado para ser representado}
            \CommentTok{# en punto fijo 1.(`nbits`-1)}
\NormalTok{            y[i, j] }\OperatorTok{=}\NormalTok{ np.clip((k }\OperatorTok{*}\NormalTok{ (}
\NormalTok{                b0 }\OperatorTok{*}\NormalTok{ y[i }\OperatorTok{-} \DecValTok{1}\NormalTok{, j] }\OperatorTok{+}
\NormalTok{                b1 }\OperatorTok{*}\NormalTok{ y[i }\OperatorTok{-} \DecValTok{1}\NormalTok{, j }\OperatorTok{-} \DecValTok{1}\NormalTok{] }\OperatorTok{+}
\NormalTok{                b2 }\OperatorTok{*}\NormalTok{ y[i }\OperatorTok{-} \DecValTok{1}\NormalTok{, j }\OperatorTok{-} \DecValTok{2}\NormalTok{] }\OperatorTok{-}
\NormalTok{                a1 }\OperatorTok{*}\NormalTok{ y[i, j }\OperatorTok{-} \DecValTok{1}\NormalTok{] }\OperatorTok{-}
\NormalTok{                a2 }\OperatorTok{*}\NormalTok{ y[i, j }\OperatorTok{-} \DecValTok{2}\NormalTok{]}
\NormalTok{            )) }\OperatorTok{>>}\NormalTok{ (iir.nbits }\OperatorTok{-} \DecValTok{1}\NormalTok{), }\DecValTok{-2}\OperatorTok{**}\NormalTok{(iir.nbits }\OperatorTok{-} \DecValTok{1}\NormalTok{), }\DecValTok{2}\OperatorTok{**}\NormalTok{(iir.nbits }\OperatorTok{-} \DecValTok{1}\NormalTok{) }\OperatorTok{-} \DecValTok{1}\NormalTok{)}
    \CommentTok{# Retorna respuesta al impulso renormalizando la salida}
    \CommentTok{# al intervalo [-1, 1)}
    \CommentTok{# salida del}
\NormalTok{    t }\OperatorTok{=}\NormalTok{ ts }\OperatorTok{*}\NormalTok{ np.arange(n)}
\NormalTok{    im }\OperatorTok{=}\NormalTok{ y[}\OperatorTok{-}\DecValTok{1}\NormalTok{, }\DecValTok{2}\NormalTok{:] }\OperatorTok{/} \DecValTok{2}\OperatorTok{**}\NormalTok{(iir.nbits }\OperatorTok{-} \DecValTok{1}\NormalTok{)}
    \ControlFlowTok{return}\NormalTok{ t, im}


\KeywordTok{def}\NormalTok{ iir2sos(iir):}
    \CommentTok{"""}
\CommentTok{    Convierte un filtro digital IIR en punto fijo a su representación}
\CommentTok{    como secuencia de secciones de segundo orden en punto flotante (ver}
\CommentTok{    `scipy.signal.sos2tf` como referencia).}

\CommentTok{    :param iir: filtro digital IIR en punto fijo.}
\CommentTok{    :return: filtro digital en representación SOS.}
\CommentTok{    """}
    \CommentTok{# Computa el límite númerico de la representación entera signada.}
\NormalTok{    n }\OperatorTok{=} \DecValTok{2}\OperatorTok{**}\NormalTok{(iir.nbits }\OperatorTok{-} \DecValTok{1}\NormalTok{)}
    \CommentTok{# Escala el filtro digital en punto fijo acorde a la ganancia y}
    \CommentTok{# normaliza al intervalo [-1, 1) en punto flotante.}
    \ControlFlowTok{return}\NormalTok{ np.array([}
\NormalTok{        (}\OperatorTok{*}\NormalTok{(sos[}\OperatorTok{-}\DecValTok{1}\NormalTok{] }\OperatorTok{*}\NormalTok{ sos[:}\DecValTok{3}\NormalTok{] }\OperatorTok{/}\NormalTok{ n), }\FloatTok{1.}\NormalTok{,}
         \OperatorTok{*}\NormalTok{(sos[}\OperatorTok{-}\DecValTok{1}\NormalTok{] }\OperatorTok{*}\NormalTok{ sos[}\DecValTok{3}\NormalTok{:}\DecValTok{5}\NormalTok{] }\OperatorTok{/}\NormalTok{ n))}
        \ControlFlowTok{for}\NormalTok{ sos }\KeywordTok{in}\NormalTok{ iir}
\NormalTok{    ])}


\KeywordTok{def}\NormalTok{ genStablePrototype(nlimit, nbits}\OperatorTok{=}\DecValTok{32}\NormalTok{):}
    \CommentTok{"""}
\CommentTok{    Genera un filtro digital IIR en punto fijo estable en}
\CommentTok{    forma aleatoria.}

\CommentTok{    :param nlimit: orden máximo admitido para el filtro.}
\CommentTok{    :param nbits: cantidad de bits utilizados para la}
\CommentTok{      representación numérica de los coeficientes.}
\CommentTok{    :return: filtro digital IIR en punto fijo generado.}
\CommentTok{    """}
\NormalTok{    iir }\OperatorTok{=}\NormalTok{ IIR()}
    \CommentTok{# Computa el límite númerico de la representación entera signada.}
\NormalTok{    n }\OperatorTok{=} \DecValTok{2} \OperatorTok{**}\NormalTok{ (nbits }\OperatorTok{-} \DecValTok{1}\NormalTok{)}
    \CommentTok{# Selecciona el orden del filtro en forma aleatoria}
    \CommentTok{# del intervalo [1, nlimit].}
\NormalTok{    order }\OperatorTok{=} \BuiltInTok{max}\NormalTok{(}\BuiltInTok{int}\NormalTok{(random.random() }\OperatorTok{*}\NormalTok{ (nlimit }\OperatorTok{+} \DecValTok{1}\NormalTok{)), }\DecValTok{1}\NormalTok{)}
    \CommentTok{# Si el orden es impar se introduce una etapa de primer orden.}
    \ControlFlowTok{if}\NormalTok{ order }\OperatorTok{%} \DecValTok{2} \OperatorTok{!=} \DecValTok{0}\NormalTok{:}
        \CommentTok{# Cero y polo de la etapa se ubican dentro o sobre el}
        \CommentTok{# círculo unidad.}
\NormalTok{        b0 }\OperatorTok{=}\NormalTok{ n}
\NormalTok{        b1 }\OperatorTok{=}\NormalTok{ np.random.randint(}\OperatorTok{-}\NormalTok{n, n}\DecValTok{-1}\NormalTok{)}
\NormalTok{        a1 }\OperatorTok{=}\NormalTok{ np.random.randint(}\OperatorTok{-}\NormalTok{n, n}\DecValTok{-1}\NormalTok{)}
\NormalTok{        sos }\OperatorTok{=}\NormalTok{ np.array([b0, b1, }\DecValTok{0}\NormalTok{, a1, }\DecValTok{0}\NormalTok{, }\DecValTok{1}\NormalTok{])}
        \CommentTok{# Ajusta la ganancia de la sección para su representación.}
\NormalTok{        fitsos(sos, nbits)}
        \CommentTok{# Incorpora la etapa al filtro.}
\NormalTok{        iir.append(sos)}
    \CommentTok{# Introduce N etapas de segundo orden para alcanzar}
    \CommentTok{# el orden seleccionado.}
    \ControlFlowTok{for}\NormalTok{ _ }\KeywordTok{in} \BuiltInTok{range}\NormalTok{(order }\OperatorTok{//} \DecValTok{2}\NormalTok{):}
        \CommentTok{# Ceros y polos de la etapa se ubican dentro del círculo unidad.}
\NormalTok{        b0 }\OperatorTok{=}\NormalTok{ n}
\NormalTok{        b2 }\OperatorTok{=}\NormalTok{ np.random.randint(}\OperatorTok{-}\NormalTok{n}\OperatorTok{+}\DecValTok{1}\NormalTok{, n}\DecValTok{-1}\NormalTok{)}
\NormalTok{        a2 }\OperatorTok{=}\NormalTok{ np.random.randint(}\OperatorTok{-}\NormalTok{n}\OperatorTok{+}\DecValTok{1}\NormalTok{, n}\DecValTok{-1}\NormalTok{)}
\NormalTok{        b1 }\OperatorTok{=}\NormalTok{ np.random.randint(}\OperatorTok{-}\NormalTok{b2}\OperatorTok{-}\NormalTok{n, b2}\OperatorTok{+}\NormalTok{n)}
\NormalTok{        a1 }\OperatorTok{=}\NormalTok{ np.random.randint(}\OperatorTok{-}\NormalTok{a2}\OperatorTok{-}\NormalTok{n, a2}\OperatorTok{+}\NormalTok{n)}
\NormalTok{        sos }\OperatorTok{=}\NormalTok{ np.array([b0, b1, b2, a1, a2, }\DecValTok{1}\NormalTok{])}
        \CommentTok{# Ajusta la ganancia de la sección para su representación.}
\NormalTok{        fitsos(sos, nbits)}
        \CommentTok{# Incorpora la etapa al filtro.}
\NormalTok{        iir.append(sos)}
    \ControlFlowTok{if} \BuiltInTok{hasattr}\NormalTok{(iir, }\StringTok{'nbits'}\NormalTok{):}
        \CommentTok{# Preserva el número de bits en el filtro.}
\NormalTok{        iir.nbits }\OperatorTok{=}\NormalTok{ nbits}
    \ControlFlowTok{return}\NormalTok{ iir}


\KeywordTok{def}\NormalTok{ cxUniformND(iir1, iir2, ndpb):}
    \CommentTok{"""}
\CommentTok{    Cruza numeradores y denominadores de filtros digitales IIR en}
\CommentTok{    punto fijo, potencialmente de distinto orden, produciendo dos}
\CommentTok{    descendientes. El orden de las etapas a cruzar es modificado}
\CommentTok{    aleatoriamente. Variante de `deap.tools.cxUniform`.}

\CommentTok{    :param iir1: primer filtro progenitor.}
\CommentTok{    :param iir2: segundo filtro progenitor.}
\CommentTok{    :param ndpb: probabilidad de cruza de numerador y/o denominador.}
\CommentTok{    """}
    \CommentTok{# Tomando el filtro candidato de menor orden, itera las}
    \CommentTok{# secciones de a pares tomados en forma aleatoria.}
    \ControlFlowTok{for}\NormalTok{ i, j }\KeywordTok{in} \BuiltInTok{zip}\NormalTok{(}
\NormalTok{        random.sample(}\BuiltInTok{list}\NormalTok{(}\BuiltInTok{range}\NormalTok{(}\BuiltInTok{len}\NormalTok{(iir1))), }\BuiltInTok{len}\NormalTok{(iir1)),}
\NormalTok{        random.sample(}\BuiltInTok{list}\NormalTok{(}\BuiltInTok{range}\NormalTok{(}\BuiltInTok{len}\NormalTok{(iir2))), }\BuiltInTok{len}\NormalTok{(iir2))}
\NormalTok{    ):}
        \CommentTok{# Obtiene las etapas de cada filtro a cruzar.}
\NormalTok{        sos1 }\OperatorTok{=}\NormalTok{ iir1[i]}
\NormalTok{        sos2 }\OperatorTok{=}\NormalTok{ iir2[j]}
        \ControlFlowTok{if}\NormalTok{ random.random() }\OperatorTok{<}\NormalTok{ ndpb:}
            \CommentTok{# Cruza los numeradores de las etapas}
\NormalTok{            sos1[:}\DecValTok{3}\NormalTok{], sos2[:}\DecValTok{3}\NormalTok{] }\OperatorTok{=}\NormalTok{ sos2[:}\DecValTok{3}\NormalTok{], sos1[:}\DecValTok{3}\NormalTok{]}
        \ControlFlowTok{if}\NormalTok{ random.random() }\OperatorTok{<}\NormalTok{ ndpb:}
            \CommentTok{# Cruza los denominadores de las etapas}
\NormalTok{            sos1[}\DecValTok{3}\NormalTok{:}\DecValTok{5}\NormalTok{], sos2[}\DecValTok{3}\NormalTok{:}\DecValTok{5}\NormalTok{] }\OperatorTok{=}\NormalTok{ sos2[}\DecValTok{3}\NormalTok{:}\DecValTok{5}\NormalTok{], sos1[}\DecValTok{3}\NormalTok{:}\DecValTok{5}\NormalTok{]}
        \CommentTok{# Ajusta la ganancia de la primera sección para que los}
        \CommentTok{# coeficientes del filtro puedan representarse en punto}
        \CommentTok{# fijo para el número de bits del filtro candidato.}
\NormalTok{        fitsos(sos1, iir1.nbits)}
        \CommentTok{# Ajusta la ganancia de la primera sección para que los}
        \CommentTok{# coeficientes del filtro puedan representarse en punto}
        \CommentTok{# fijo para el número de bits del filtro candidato.}
\NormalTok{        fitsos(sos2, iir2.nbits)}
    \ControlFlowTok{return}\NormalTok{ iir1, iir2}


\KeywordTok{def}\NormalTok{ evTimeResponse(iir, target, ts):}
    \CommentTok{"""}
\CommentTok{    Evalúa la aptitud de la respuesta temporal de un filtro}
\CommentTok{    digital IIR en punto fijo según la similitud que su}
\CommentTok{    respuesta temporal presenta respecto a la respuesta}
\CommentTok{    objetivo.}

\CommentTok{    :param iir: filtro digital IIR en punto fijo.}
\CommentTok{    :param target: respuesta al impulso objetivo.}
\CommentTok{    :param ts: período de muestreo, en segundos.}
\CommentTok{    :return: aptitud del filtro provisto.}
\CommentTok{    """}
    \CommentTok{# Computa la respuesta al impulso del filtro candidato}
    \CommentTok{# en su representación SOS.}
\NormalTok{    _, (im,) }\OperatorTok{=}\NormalTok{ signal.dimpulse(}
\NormalTok{        (}\OperatorTok{*}\NormalTok{signal.sos2tf(iir2sos(iir)), ts), n}\OperatorTok{=}\BuiltInTok{len}\NormalTok{(target)}
\NormalTok{    )}
    \CommentTok{# Computa el error relativo entre respuesta al impulso}
    \CommentTok{# del filtro candidato y respuesta al impulso esperada.}
\NormalTok{    et }\OperatorTok{=}\NormalTok{ (im }\OperatorTok{-}\NormalTok{ target) }\OperatorTok{/}\NormalTok{ np.}\BuiltInTok{max}\NormalTok{(np.}\BuiltInTok{abs}\NormalTok{(target))}
    \CommentTok{# Evalua la aptitud del filtro candidato como el recíproco}
    \CommentTok{# de la potencia de error relativo.}
    \ControlFlowTok{return}\NormalTok{ (}\FloatTok{1.} \OperatorTok{/}\NormalTok{ (np.mean(et)}\OperatorTok{**}\DecValTok{2} \OperatorTok{+}\NormalTok{ np.var(et)),)}


\KeywordTok{def}\NormalTok{ mutCoeffGaussian(iir, mu, sigma, indpb):}
    \CommentTok{"""}
\CommentTok{    Muta los coeficientes de un filtro digital IIR en punto}
\CommentTok{    fijo mediante perturbaciones numéricas. Variante de}
\CommentTok{    `deap.tools.mutGaussian`.}

\CommentTok{    :param mu: media de la distribución gaussiana de la que se toman}
\CommentTok{       las perturbaciones a aplicar.}
\CommentTok{    :param sigma: desvío estandar de la distribución gaussiana de la}
\CommentTok{       que se toman las perturbaciones a aplicar.}
\CommentTok{    :param indpb: probabilidad de perturbar un coeficiente.}
\CommentTok{    """}
    \CommentTok{# Itera cada sección del filtro.}
    \ControlFlowTok{for}\NormalTok{ sos }\KeywordTok{in}\NormalTok{ iir:}
        \CommentTok{# Conforma una máscara de los coeficientes de la}
        \CommentTok{# sección actual del filtro, según la probabilidad}
        \CommentTok{# dada.}
\NormalTok{        mask }\OperatorTok{=}\NormalTok{ (np.random.random(}\BuiltInTok{len}\NormalTok{(sos)}\OperatorTok{-}\DecValTok{1}\NormalTok{) }\OperatorTok{<}\NormalTok{ indpb)}
        \CommentTok{# Perturba los coeficientes a partir de una distribución}
        \CommentTok{# normal con media y desvío estándar dados.}
\NormalTok{        sos[:}\OperatorTok{-}\DecValTok{1}\NormalTok{][mask] }\OperatorTok{+=}\NormalTok{ np.random.normal(}
\NormalTok{            mu, sigma, np.count_nonzero(mask)}
\NormalTok{        ).astype(}\BuiltInTok{int}\NormalTok{)}
        \CommentTok{# Ajusta la ganancia de la sección para que los coeficientes}
        \CommentTok{# del filtro puedan representarse en punto fijo para el}
        \CommentTok{# número de bits del filtro.}
\NormalTok{        fitsos(sos, iir.nbits)}
    \ControlFlowTok{return}\NormalTok{ iir,}


\KeywordTok{def}\NormalTok{ eaSimplePlusElitism(population, toolbox, cxpb, mutpb, eprop, ngen,}
\NormalTok{                        stats}\OperatorTok{=}\VariableTok{None}\NormalTok{, halloffame}\OperatorTok{=}\VariableTok{None}\NormalTok{, verbose}\OperatorTok{=}\VariableTok{__debug__}\NormalTok{):}
    \CommentTok{"""}
\CommentTok{    Variante de `deap.algorithms.eaSimple` con una proporción de elitismo.}
\CommentTok{    """}
\NormalTok{    logbook }\OperatorTok{=}\NormalTok{ tools.Logbook()}
\NormalTok{    logbook.header }\OperatorTok{=}\NormalTok{ [}\StringTok{'gen'}\NormalTok{, }\StringTok{'nevals'}\NormalTok{] }\OperatorTok{+}\NormalTok{ (stats.fields }\ControlFlowTok{if}\NormalTok{ stats }\ControlFlowTok{else}\NormalTok{ [])}

    \CommentTok{# Evalua los individuos con aptitud inválida.}
\NormalTok{    invalid_ind }\OperatorTok{=}\NormalTok{ [ind }\ControlFlowTok{for}\NormalTok{ ind }\KeywordTok{in}\NormalTok{ population }\ControlFlowTok{if} \KeywordTok{not}\NormalTok{ ind.fitness.valid]}
\NormalTok{    fitnesses }\OperatorTok{=}\NormalTok{ toolbox.}\BuiltInTok{map}\NormalTok{(toolbox.evaluate, invalid_ind)}
    \ControlFlowTok{for}\NormalTok{ ind, fit }\KeywordTok{in} \BuiltInTok{zip}\NormalTok{(invalid_ind, fitnesses):}
\NormalTok{        ind.fitness.values }\OperatorTok{=}\NormalTok{ fit}

    \ControlFlowTok{if}\NormalTok{ halloffame }\KeywordTok{is} \KeywordTok{not} \VariableTok{None}\NormalTok{:}
\NormalTok{        halloffame.update(population)}

\NormalTok{    record }\OperatorTok{=}\NormalTok{ stats.}\BuiltInTok{compile}\NormalTok{(population) }\ControlFlowTok{if}\NormalTok{ stats }\ControlFlowTok{else}\NormalTok{ \{\}}
\NormalTok{    logbook.record(gen}\OperatorTok{=}\DecValTok{0}\NormalTok{, nevals}\OperatorTok{=}\BuiltInTok{len}\NormalTok{(invalid_ind), }\OperatorTok{**}\NormalTok{record)}
    \ControlFlowTok{if}\NormalTok{ verbose:}
        \BuiltInTok{print}\NormalTok{(logbook.stream)}

    \CommentTok{# Comienza el proceso evolutivo.}
    \ControlFlowTok{for}\NormalTok{ gen }\KeywordTok{in} \BuiltInTok{range}\NormalTok{(}\DecValTok{1}\NormalTok{, ngen }\OperatorTok{+} \DecValTok{1}\NormalTok{):}
        \CommentTok{# Seleccina la próxima generación de individuos.}
\NormalTok{        offspring }\OperatorTok{=}\NormalTok{ toolbox.select(population, }\BuiltInTok{len}\NormalTok{(population))}

        \CommentTok{# Varia el pool de individuos, aplicando cruza y mutación.}
\NormalTok{        offspring }\OperatorTok{=}\NormalTok{ algorithms.varAnd(offspring, toolbox, cxpb, mutpb)}

        \CommentTok{# Evalua los individuos con aptitud inválida.}
\NormalTok{        invalid_ind }\OperatorTok{=}\NormalTok{ [ind }\ControlFlowTok{for}\NormalTok{ ind }\KeywordTok{in}\NormalTok{ offspring }\ControlFlowTok{if} \KeywordTok{not}\NormalTok{ ind.fitness.valid]}
\NormalTok{        fitnesses }\OperatorTok{=}\NormalTok{ toolbox.}\BuiltInTok{map}\NormalTok{(toolbox.evaluate, invalid_ind)}
        \ControlFlowTok{for}\NormalTok{ ind, fit }\KeywordTok{in} \BuiltInTok{zip}\NormalTok{(invalid_ind, fitnesses):}
\NormalTok{            ind.fitness.values }\OperatorTok{=}\NormalTok{ fit}

        \CommentTok{# Actualiza el grupo de mejores individuos.}
        \ControlFlowTok{if}\NormalTok{ halloffame }\KeywordTok{is} \KeywordTok{not} \VariableTok{None}\NormalTok{:}
\NormalTok{            halloffame.update(offspring)}

        \CommentTok{# Reemplaza la población actual con los mejores del conjunto}
        \CommentTok{# compuesta por su descendencia y la elite.}
\NormalTok{        elite_count }\OperatorTok{=} \BuiltInTok{int}\NormalTok{(}\BuiltInTok{len}\NormalTok{(population) }\OperatorTok{*}\NormalTok{ eprop)}
\NormalTok{        elite }\OperatorTok{=}\NormalTok{ tools.selBest(population, elite_count)}
\NormalTok{        population[:] }\OperatorTok{=}\NormalTok{ tools.selBest(offspring }\OperatorTok{+}\NormalTok{ elite, }\BuiltInTok{len}\NormalTok{(population))}

        \CommentTok{# Toma nota de las estadísticas de la generación actual.}
\NormalTok{        record }\OperatorTok{=}\NormalTok{ stats.}\BuiltInTok{compile}\NormalTok{(population) }\ControlFlowTok{if}\NormalTok{ stats }\ControlFlowTok{else}\NormalTok{ \{\}}
\NormalTok{        logbook.record(gen}\OperatorTok{=}\NormalTok{gen, nevals}\OperatorTok{=}\BuiltInTok{len}\NormalTok{(invalid_ind), }\OperatorTok{**}\NormalTok{record)}
        \ControlFlowTok{if}\NormalTok{ verbose:}
            \BuiltInTok{print}\NormalTok{(logbook.stream)}

    \ControlFlowTok{return}\NormalTok{ population, logbook}


\KeywordTok{def}\NormalTok{ configure_genetic_approx(}\OperatorTok{*}\NormalTok{, nbits}\OperatorTok{=}\DecValTok{16}\NormalTok{, nlimit}\OperatorTok{=}\DecValTok{8}\NormalTok{, nsln}\OperatorTok{=}\DecValTok{10}\NormalTok{, cxpb}\OperatorTok{=}\FloatTok{0.7}\NormalTok{, ndpb}\OperatorTok{=}\FloatTok{0.5}\NormalTok{,}
\NormalTok{                             mutpb}\OperatorTok{=}\FloatTok{0.2}\NormalTok{, mutmean}\OperatorTok{=}\FloatTok{0.0}\NormalTok{, mutstd}\OperatorTok{=}\FloatTok{0.3}\NormalTok{, coeffpb}\OperatorTok{=}\FloatTok{0.1}\NormalTok{,}
\NormalTok{                             tournsize}\OperatorTok{=}\DecValTok{3}\NormalTok{, nind}\OperatorTok{=}\DecValTok{1000}\NormalTok{, eprop}\OperatorTok{=}\FloatTok{0.005}\NormalTok{, ngen}\OperatorTok{=}\DecValTok{400}\NormalTok{,}
\NormalTok{                             verbose}\OperatorTok{=}\VariableTok{__debug__}\NormalTok{):}
    \CommentTok{"""}
\CommentTok{    Configura una función de aproximación de filtros digitales IIR en punto}
\CommentTok{    fijo con una dada respuesta al impulso.}

\CommentTok{    :param nbits: cantidad de bits utilizados para la representación numérica}
\CommentTok{      de los coeficientes la solución.}
\CommentTok{    :param nlimit: orden máximo admitido para la solución.}
\CommentTok{    :param nsln: cantidad de soluciones a conservar de entre las más aptas.}
\CommentTok{    :param cxpb: probabilidad de cruza genética de soluciones.}
\CommentTok{    :param ndpb: probabilidad de intercambio de numerador y denominador}
\CommentTok{      de soluciones seleccionadas para la cruza.}
\CommentTok{    :param mutpb: probabilidad de mutar una solución.}
\CommentTok{    :param mutmean: media de perturbaciones utilizadas para la mutación,}
\CommentTok{      en el intervalo [-1, 1).}
\CommentTok{    :param mutstd: desvío estándar de las perturbaciones utilizadas para}
\CommentTok{      la mutación, en el intervalo [-1, 1).}
\CommentTok{    :param coeffpb: probabilidad de perturbación de un coeficiente de}
\CommentTok{      una solución seleccionada para la mutación.}
\CommentTok{    :param tournsize: cantidad de soluciones a someter en cada instancia}
\CommentTok{      de torneo de selección.}
\CommentTok{    :param nind: cantidad de soluciones en el pool genético a explorar.}
\CommentTok{    :param eprop: proporción de elitismo o la cantidad de soluciones de elite}
\CommentTok{      respecto al total de soluciones en el pool.}
\CommentTok{    :param ngen: cantidad de generaciones a evolucionar.}
\CommentTok{    :return: función de aproximación de filtro digital IIR en punto fijo.}
\CommentTok{    """}
    \KeywordTok{def}\NormalTok{ approx(target, ts, nlimit}\OperatorTok{=}\NormalTok{nlimit, nsln}\OperatorTok{=}\NormalTok{nsln):}
        \CommentTok{"""}
\CommentTok{        Aproxima un filtro digital IIR en punto fijo para que presente la}
\CommentTok{        respuesta al impulso objetivo.}

\CommentTok{        :param target: respuesta al impulso objetivo.}
\CommentTok{        :param ts: período de muestreo, en segundos.}
\CommentTok{        :param nlimit: orden máximo admitido para la solución.}
\CommentTok{        :param nsln: cantidad de soluciones a conservar de entre las más aptas.}
\CommentTok{        :return: filtro digital IIR en punto fijo.}
\CommentTok{        """}
\NormalTok{        toolbox }\OperatorTok{=}\NormalTok{ base.Toolbox()}
\NormalTok{        toolbox.register(}
            \StringTok{'individual'}\NormalTok{, genStablePrototype, nlimit}\OperatorTok{=}\NormalTok{nlimit, nbits}\OperatorTok{=}\NormalTok{nbits}
\NormalTok{        )}
\NormalTok{        toolbox.register(}
            \StringTok{'population'}\NormalTok{, tools.initRepeat, }\BuiltInTok{list}\NormalTok{, toolbox.individual}
\NormalTok{        )}
\NormalTok{        toolbox.register(}\StringTok{'mate'}\NormalTok{, cxUniformND, ndpb}\OperatorTok{=}\NormalTok{ndpb)}
\NormalTok{        toolbox.register(}\StringTok{'select'}\NormalTok{, tools.selTournament, tournsize}\OperatorTok{=}\NormalTok{tournsize)}
\NormalTok{        toolbox.register(}
            \StringTok{'mutate'}\NormalTok{, mutCoeffGaussian, mu}\OperatorTok{=}\NormalTok{mutmean}\OperatorTok{*}\DecValTok{2}\OperatorTok{**}\NormalTok{(nbits}\DecValTok{-1}\NormalTok{),}
\NormalTok{            sigma}\OperatorTok{=}\NormalTok{mutstd}\OperatorTok{*}\DecValTok{2}\OperatorTok{**}\NormalTok{(nbits}\DecValTok{-1}\NormalTok{), indpb}\OperatorTok{=}\NormalTok{coeffpb}
\NormalTok{        )}
\NormalTok{        toolbox.register(}\StringTok{'evaluate'}\NormalTok{, evTimeResponse, target}\OperatorTok{=}\NormalTok{target, ts}\OperatorTok{=}\NormalTok{ts)}

\NormalTok{        stats }\OperatorTok{=}\NormalTok{ tools.Statistics(}
            \KeywordTok{lambda}\NormalTok{ individual: individual.fitness.values}
\NormalTok{        )}
\NormalTok{        stats.register(}\StringTok{'mean_fitness'}\NormalTok{, np.mean)}
\NormalTok{        stats.register(}\StringTok{'fitness_stddev'}\NormalTok{, np.std)}
\NormalTok{        stats.register(}\StringTok{'min_fitness'}\NormalTok{, np.}\BuiltInTok{min}\NormalTok{)}
\NormalTok{        stats.register(}\StringTok{'max_fitness'}\NormalTok{, np.}\BuiltInTok{max}\NormalTok{)}

\NormalTok{        hall }\OperatorTok{=}\NormalTok{ tools.HallOfFame(}
\NormalTok{            maxsize}\OperatorTok{=}\NormalTok{nsln, similar}\OperatorTok{=}\KeywordTok{lambda}\NormalTok{ x, y: (}
\NormalTok{                np.}\BuiltInTok{all}\NormalTok{(np.equal(np.shape(x), np.shape(y)))}
                \KeywordTok{and}\NormalTok{ np.}\BuiltInTok{all}\NormalTok{(np.equal(x, y))}
\NormalTok{            )}
\NormalTok{        )}

\NormalTok{        population }\OperatorTok{=}\NormalTok{ toolbox.population(nind)}
\NormalTok{        offspring, logbook }\OperatorTok{=}\NormalTok{ eaSimplePlusElitism(}
\NormalTok{            population, toolbox, cxpb}\OperatorTok{=}\NormalTok{cxpb, mutpb}\OperatorTok{=}\NormalTok{mutpb,}
\NormalTok{            eprop}\OperatorTok{=}\NormalTok{eprop, ngen}\OperatorTok{=}\NormalTok{ngen, stats}\OperatorTok{=}\NormalTok{stats, halloffame}\OperatorTok{=}\NormalTok{hall,}
\NormalTok{            verbose}\OperatorTok{=}\NormalTok{verbose}
\NormalTok{        )}

        \ControlFlowTok{return}\NormalTok{ hall, offspring, logbook}
    \ControlFlowTok{return}\NormalTok{ approx}
\end{Highlighting}
\end{Shaded}

\hrulefill

    \subsubsection{Desempeño}\label{desempeuxf1o}

En lo siguiente, se aproximan las respuestas impulsionales de filtros
aproximados analíticamente y se contrastan los resultados obtenidos por
el algoritmo para un filtro IIR en punto fijo 1.15 (típico Q15 de 16
bits) con la solución cerrada. En cada una de las comparaciones que
siguen, los siguientes parámetros permanecen constantes:

\begin{longtable}[]{@{}ll@{}}
\toprule
Parámetro & Valor\tabularnewline
\midrule
\endhead
Máximo orden del filtro & 10\tabularnewline
Frecuencia de sampleo & 44 kHz\tabularnewline
Número de bits & 16\tabularnewline
Número de individuos & 500\tabularnewline
Tamaño de torneo de individuos & 3\tabularnewline
Probabilidad de cruza de individuos & 0.7\tabularnewline
Probabilidad de cruza de num. y den. & 0.5\tabularnewline
Probabilidad de mutación de individuos & 0.2\tabularnewline
Probabilidad de mutación de coeficiente & 0.1\tabularnewline
Media de perturbación & 0.0\tabularnewline
Desvío estándar de perturbación & 0.3\tabularnewline
Número de generaciones & 500\tabularnewline
Proporción de elitismo & 0.5\%\tabularnewline
Tamaño de historial de individuos & 10\tabularnewline
\bottomrule
\end{longtable}


    \paragraph{Filtro pasa bajo tipo
Butterworth}\label{filtro-pasa-bajo-tipo-butterworth}

La aproximación de Butterworth logra máxima planicidad de módulo de
respuesta en la banda de paso, a costa de menor pendiente en la banda de
transición hacia la banda de paso. Para el caso del filtro pasa bajo en
tiempo continuo, esta aproximación puede describirse mediante:

\[
\|H(\omega)\|^2 = \frac{1}{\sqrt{1 + \epsilon^2 \frac{\omega}{\omega_c}^2}}
\]

De \(\omega\) se mapea a \(s = \sigma + j\omega\), para luego arribar a
\(z\) vía transformada bilineal
\(s = \frac{2}{t_s} \frac{z - 1}{z + 1}\) y posterior correción por
\emph{frequency warping}.

Se diseña entonces un filtro pasa bajos con atenuación de
\(\leq 1 dB @ 6.6 kHz\) y de \(\geq 40 dB @ 13.2 kHz\), para luego
utilizar la respuesta impulsional resultante para aproximar la
implementación con filtro IIR en punto fijo \(1.15\).


    \begin{center}
    \adjustimage{max size={0.9\linewidth}{0.9\paperheight}}{output_23_0.png}
    \end{center}
    { \hspace*{\fill} \\}
    

    \begin{center}
    \adjustimage{max size={0.9\linewidth}{0.9\paperheight}}{output_24_0.png}
    \end{center}
    { \hspace*{\fill} \\}
    

    \begin{center}
    \adjustimage{max size={0.9\linewidth}{0.9\paperheight}}{output_25_0.png}
    \end{center}
    { \hspace*{\fill} \\}
    
    \paragraph{Filtro pasa alto tipo
Chebyshev}\label{filtro-pasa-alto-tipo-chebyshev}

La aproximación de Chebyshev minimiza el error respecto a la plantilla
de módulo de atenuación a costa de ripple en la banda de paso. Para el
caso del filtro pasa bajo en tiempo continuo, esta aproximación puede
describirse mediante:

\[
\|H(\omega)\|^2 = \frac{1}{\sqrt{1 + \epsilon^2 T^2_n(\frac{\omega}{\omega_c})}}
\]

donde


\begin{align}
T_0(x) &= 0 \\
T_1(x) &= 1 \\
T_{n+1}(x) &= 2x T_n(x) - T_{n-1}(x)
\end{align}


son los polinomios de Chebyshev del primer tipo.

De \(\omega\) se mapea a \(s = \sigma + j\omega\) y se aplica el kernel
de transformación \(s = \frac{1}{s'}\), para luego arribar a \(z\) vía
transformada bilineal \(s' = \frac{2}{t_s} \frac{z - 1}{z + 1}\) y
posterior correción por \emph{frequency warping}.

Se diseña entonces un filtro pasa altos con atenuación de
\(\geq 40 dB @ f = 6.6 kHz\) y de \(\leq 3 dB @ f = 13.2 kHz\), para
luego utilizar la respuesta impulsional resultante para aproximar la
implementación con filtro IIR en punto fijo \(1.15\).


    \begin{center}
    \adjustimage{max size={0.9\linewidth}{0.9\paperheight}}{output_28_0.png}
    \end{center}
    { \hspace*{\fill} \\}
    

    \begin{center}
    \adjustimage{max size={0.9\linewidth}{0.9\paperheight}}{output_29_0.png}
    \end{center}
    { \hspace*{\fill} \\}
    

    \begin{center}
    \adjustimage{max size={0.9\linewidth}{0.9\paperheight}}{output_30_0.png}
    \end{center}
    { \hspace*{\fill} \\}
    
    \paragraph{Filtro pasa banda tipo Chebyshev
Inverso}\label{filtro-pasa-banda-tipo-chebyshev-inverso}

La aproximación de Chebyshev inversa no presenta ripple en la banda de
paso sino en la banda de atenuación, a costa de una pendiente en la
banda de transición. Para el caso del filtro pasa bajo en tiempo
continuo, esta aproximación puede describirse mediante:

\[
\|H(\omega)\|^2 = \frac{1}{\sqrt{1 + \frac{1}{\epsilon^2 T^2_n(\frac{\omega}{\omega_c})}}}
\]

donde


\begin{align}
T_0(x) &= 0 \\
T_1(x) &= 1 \\
T_{n+1}(x) &= 2x T_n(x) - T_{n-1}(x)
\end{align}


son los polinomios de Chebyshev del primer tipo.

De \(\omega\) se mapa a \(s = \sigma + j\omega\) y se aplica el kernel
de transformación \(s = Q (\frac{s'}{\omega_o} + \frac{\omega_o}{s'})\),
para luego arribar a \(z\) vía transformada bilineal
\(s' = \frac{2}{t_s} \frac{z - 1}{z + 1}\) y posterior correción por
\emph{frequency warping}.

Se diseña entonces un filtro pasa banda con atenuación
\(\leq 1 dB @ 6.6 kHz < f < 8.8 kHz\) y
\(\geq 40 dB @ f > 13.2 kHz \lor f < 2.2 kHz\), para luego utilizar la
respuesta impulsional resultante para aproximar la implementación con
filtro IIR en punto fijo \(1.15\).


    \begin{center}
    \adjustimage{max size={0.9\linewidth}{0.9\paperheight}}{output_33_0.png}
    \end{center}
    { \hspace*{\fill} \\}
    

    \begin{center}
    \adjustimage{max size={0.9\linewidth}{0.9\paperheight}}{output_34_0.png}
    \end{center}
    { \hspace*{\fill} \\}
    

    \begin{center}
    \adjustimage{max size={0.9\linewidth}{0.9\paperheight}}{output_35_0.png}
    \end{center}
    { \hspace*{\fill} \\}
    
    \paragraph{Filtro multi banda tipo FIR con ventana
Hamming}\label{filtro-multi-banda-tipo-fir-con-ventana-hamming}

Los filtros FIR o de respuesta impulsional finita pueden ser descriptos
empleando la ecuación en diferencias de los filtros IIR con
\(a_0 = 1 \land a_m = 0\ \forall m \neq 0\). Para su diseño, el uso del
método de ventana parte de una plantilla de módulo de respuesta en
frecuencia ideal o \emph{brickwall}, simétrica respecto a las
frecuencias negativas. La respuesta impulsional asociada es una
sumatoria de senos cardinales, ergo no causal ni finita. Se procede
entonces a la aplicación de una ventana de morfología apropiada que
acote temporalmente dicha respuesta.

\[
h_w[k] = h[k] \cdot w[k]
\]

Una de muchas ventanas es la de Hamming:

\[
w[k] = 0.53836 - 0.46164 \cos\Bigl(\frac{2 \pi k}{N - 1}\Bigr)
\]

donde \(0 \leq k < N\). Esta ventana es similar una ventana de coseno
elevado pero, a diferencia de ésta, cancela el primer lóbulo de la
respuesta.

Se diseña entonces un filtro multi banda con \(T = 64\) coeficientes o
\emph{taps} y bandas de paso \(0 Hz \leq f < 3 KHz\) y
\(10 KHz \leq f < 12 kHz\), para luego utlizar la respuesta impulsional
resultante para aproximar la implementación con filtro IIR en punto fijo
\(1.15\).



    \begin{center}
    \adjustimage{max size={0.9\linewidth}{0.9\paperheight}}{output_38_0.png}
    \end{center}
    { \hspace*{\fill} \\}
    

    \begin{center}
    \adjustimage{max size={0.9\linewidth}{0.9\paperheight}}{output_39_0.png}
    \end{center}
    { \hspace*{\fill} \\}
    

    \begin{center}
    \adjustimage{max size={0.9\linewidth}{0.9\paperheight}}{output_40_0.png}
    \end{center}
    { \hspace*{\fill} \\}
    
    \subsection{Conclusiones}\label{conclusiones}

Se ha descripto un algoritmo capaz de aproximar respuestas
impulsionales, y con menor exactitud la respuesta en frecuencia
asociada, en filtros digitales IIR en punto fijo como cascada de
secciones bicuadráticas.

En términos de desempeño, el conjunto se mantiene estable a pesar de las
limitaciones de la representación numérica y la artimética asociada.
Como contrapartida, la fase de la respuesta en frecuencia resultante
guarda poca o ninguna relación con la plantilla y el módulo exhibe
notables errores (en escala logarítmica), en especial en las
inmediaciones de los ceros. Se observa, además, la tendencia del
algoritmo a converger a respuestas impulsionales constantes según la
plantilla, probablemente atribuible a la función de aptitud y a la
superficie de error que impone. Ésto restringe considerablemente la
aplicación de este algoritmo, por lo que es preciso profundizar en las
causas o en última instancia modificar el enfoque, ya sea acotando la
población inicial a las inmediaciones de soluciones analíticas o
utilizando la población final como punto de partida para otra instancia
de aproximación.

En términos de aplicación, al requerir una respuesta impulsional para la
aproximación se supone más información que la que usualmente se
encuentra disponible. La aproximación directa de módulo o fase de
respuesta en frecuencia es típica y este algoritmo no ha mostrado buenos
resultados con funciones de aptitud basadas en métricas afines. La
derivación de una respuesta impulsional a partir de esta información es
un posible camino a seguir, con sus propios desafíos.

Finalmente, existe una amplia variedad de algoritmos inspirados en la
naturaleza e incluso híbridos con otros tipos de técnicas de
optimización que aún restan por explorar. Cabe resaltar, no obstante, la
cuidadosa definición de la representación y los operadores genéticos,
adecuados al universo de discurso, y el mantenimiento de un delicado
balance entre diversidad genética y presión de selección que este (tipo
de) algoritmo requiere para converger a soluciones cuando mucho
subóptimas sin tornarse una búsqueda por fuerza bruta aleatoria.

    \subsection{Referencias}\label{referencias}

{[}1{]} R. Lyons, Understanding digital signal processing. Upper Saddle
River {[}etc.{]}: Pearson Education International, 2013.

{[}2{]} A. EIBEN, Introduction To Evolutionary Computing:
SPRINGER-VERLAG BERLIN AN, 2016.


    % Add a bibliography block to the postdoc
    
    
    
    \end{document}
